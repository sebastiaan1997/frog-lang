\documentclass{report}


\begin{document}

\chapter{Introduction}
The goal of the language is to write code efficiently and easally while it's still extendable, even at the core language.
The language is thereby inspired by Rust, C\#, Python and C/C++.


\chapter{Keywords}

\section{Variables}
\paragraph{const} Defines a constant
\paragraph{let} Defines a variable

\section{Flow Control keywords}

\subsection {Conditional}

\paragraph{if} Indicates an if statement
\paragraph{else} Indicates an else clause of an if statement
\paragraph{when} Indicates pattern-matching construct



\subsection{Procedures}
\paragraph{fn} Function, a pure function, without side-effects
\paragraph{rt} Routine, a not pure function, may have side-effects

\section{Structures}
\paragraph{struct} Indicates a datastructure
\paragraph{enum} Indicates an enumeration


\section {Other}

\paragraph{infix} Indicates that a function can be placed between 2 constants or variables.
\paragraph{assoc} Indicates that a procedure is associated with a structure.


\chapter{Operators}

\section{Logical}

\paragraph{$||$} Logical or
\paragraph{$\&\&$} Logical and
\paragraph{$!$} not
\section{Assignment}
\paragraph{$=$} Assigns a certain value to a variable or constant.

\subsection{Combined}






\begin{itemize}
    \item{+}
    \item{-}
    \item{*}
    \item{/}
    \item{}
\end{itemize}


\chapter{Grammar}
\section{Comments}
\begin{verbatim}
    // I'm a single line comment

    /*
     * I'm a multi-line 
     * comment
     */function
\end{verbatim}
\subsection{Variables}
\begin{verbatim}
    let a = 0;
    let b, c = 1;
    let d = 2, e = 3;
\end{verbatim}
\subsection{Constants}
Constants are immutable variables.
\begin{verbatim}
    const a = 0;
    const b, c = 1;
    const d = 2, e = 3;
\end{verbatim}


\section{Flow control}

\subsection{if-statement}
\begin{verbatim}
    if a == b {
        // code...
    }
\end{verbatim}


\subsection {Pattern match}

\begin{verbatim}
when x {
    0 => print("Hello World");
    1 => print("Hello John");
}
\end{verbatim}



\subsection{Veriable declaration}

\begin{verbatim}
    fn hello(x: isize)
        const a = when x {
            0 => "Aap";
            1 => "Noot";
            2 => "Mies";
        }
        print(a);
    }
\end{verbatim}


\subsection{Variable assignment}

\section {typing}
\subsection {Structs}
\begin{verbatim}
    struct Structure {
        a, b: int;
        c: float;
    }
    \end{verbatim}

\subsection {Enums}
\begin{verbatim}
    enum Seasons {
        winter,
        spring
        summer,
        autumn
    }
\end{verbatim}

\section{Literals}
\subsection {Strings}


\section {Routines}

\subsection {Callables}
Callables are functions or routines that can be called from somewhere else.
\begin{verbatim}
    rt print_hello(){
        print("Hello")
    }

\end{verbatim}
\subsection {Functions}
Functions cannot, so do not have side-effects
\begin{verbatim}
    fn sum(lhs: int, rhs: int) {
        return a + b;
    }

\end{verbatim}

\subsection{Shorthand notation}
\begin{verbatim}
    rt print_hello => print("Hello");
    fn sum(a: int, b: int) => a + b;
\end{verbatim}


\subsection {Infix}
\begin{verbatim}
    infix fn + (a, b)
\end{verbatim}


\chapter{Examples}
\section{Hello World}
\begin{verbatim}
    rt main() => print("Hello World!");
\end{verbatim}

\section{Operator Overloading}
Operator overloading can be accheived by defining the operator as an Infix callable.
\begin{verbatim}
    struct Vector [int:2];
    infix fn + (rhs: Vector, lhs: Vector) => [ lhs[0] + rhs[0], lhs[1] + rhs[1] ]

    rt main() {
        const a = Vector[1, 2];
        const b = Vector[3, 4];
        print(a + b); // Vector[4, 6]
    }
\end{verbatim}

\section {Custom operators}

\begin{verbatim}
    infix fn % (rhs: int, lhs: int) => (rhs / lhs) + rhs - (rhs / lhs);

    rt main() {
        print(5 % 2); // 1
    }
\end{verbatim}


\section{Member callable declaration}

\begin{verbatim}
    struct Vector [int:2];
    member rt Vector::print() => print(print[0], print[1]);

    rt main() {
        const a = Vector[1, 2];
        const b = Vector[3, 4];
        print(a + b); // Vector[4, 6]
    }
\end{verbatim}

\end{document}